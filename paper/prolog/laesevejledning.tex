\documentclass[../main.tex]{subfiles}

\begin{document}
\section*{Læsevejledning}
\addcontentsline{toc}{section}{Læsevejledning}

Rapporten anbefales at blive læst kronologisk, da man derved får den bedste forståelse for arbejdsprocessen i projektet. Det er dog muligt at læse de enkelte afsnit uafhængigt af hinanden, men hvis der refereres til andre afsnit, anbefales det at læse dette afsnit for at opnå den fulde forståelse. 

Der bliver igennem rapporten brugt en speciel syntaks når der snakkes om begreber fra koden. Den ser ud som følger: \code{Eksempel på syntaks}. Alle referencer er hyperlinks, hvilket vil sige at det er muligt at trykke på referencen, og derved blive ført til det refererede punkt i rapporten.  

Store dele af projektetarbejdet er foregået igennem GitHub. Her kan blandt andet koden til projektet findes, ved brug af linket: \url{https://github.com/kvartborg/2-semesterprojekt} . For at kunne teste programmet skal der logges ind. Her kan der bruges brugeren: Username: admin Password: admin .
Kodebasen afleveres med en allerede opsat \code{PostgreSQLConfig} fil, forbundet til en oprettet PostgreSQL database, der både er struktureret og populeret med dummy data. Det skulle altså være muligt at køre projektet direkte efter udpakning af zip-filen.

\newpage
\end{document}
