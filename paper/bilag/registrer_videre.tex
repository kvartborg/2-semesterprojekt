\documentclass[../../main.tex]{subfiles}

\begin{document}

\tocless\subsubsection{Registrer Videre forløb (udvider DB-0000 Sagsåbning)}

\begin{table}[H]
\centering
\resizebox{0.9 \textwidth}{!}{%
\begin{tabular}{| p{1\textwidth} |} \hline
Brugsmønster: VidereForløb (udvider DB-0000 Sagsåbning)   \\ \hline

ID: DB-0006 \\ \hline

\textbf{Kort beskrivelse:} \\
Sagsbehandler noterer hvilke aftaler der er indgået med borgeren om det videre forløb.       \\ \hline

\textbf{Primære aktører:} \\
Sagsbehandler \\ \hline

\textbf{Sekundære  aktører:} \\
Borger og Pårørende  \\ \hline

\textbf{Prækondition:} \\
  \begin{minipage}[t]{\textwidth}
    \begin{itemize}
    \item[-] Sagen er blevet registreret, samt basis oplysninger er blevet indhentet og evt, særlige forhold er blevet noteret. Når disse oplysninger er indtastet oprettes sagen endeligt og gemmes. \\
    \end{itemize}
  \end{minipage} \\ \hline

\textbf{Hovedhændelsesforløb:} \\
  \begin{minipage}[t]{\textwidth}
    \begin{enumerate}
    \item Brugsmønsteret starter når sagsbehandler noterer hvilke aftaler der er indgået med borgeren om det videre forløb.
    \item Sagsbehandleren trykker gem, og sagen gemmes i persistent. \\
    \end{enumerate}
  \end{minipage} \\ \hline

\textbf{Postkondition:} \\
  \begin{minipage}[t]{\textwidth}
    \begin{itemize}
    \item[-] Aftaler omkring det videre forløb er blevet noteret på sagen.
    \item[-] Sagen er gemt permanent. \\
    \end{itemize}
  \end{minipage} \\ \hline

\textbf{Alternative hændelsesforløb:} \\
  \begin{minipage}[t]{\textwidth}
    \begin{itemize}
    \item[-]
    \end{itemize}
  \end{minipage} \\ \hline
 
\end{tabular}}
\caption{VidereForløb}
\label{db:006}
\end{table}

\pagebreak{}

\end{document}