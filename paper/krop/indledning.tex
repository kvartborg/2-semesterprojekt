\documentclass[../main.tex]{subfiles}

\begin{document}

\section{Indledning} \label{indledning}
I forbindelse med monopolbruddet i 2008, og den nye økonomiaftale i 2013, blev der åbnet et nyt marked for landsdækkende IT-løsninger til det offentlige. Det er dette marked, EG Team Online ønsker at træde ind på, ved udviklingen af Sensum Udred. Sensum Udred er et sagsbehandlingssystem til voksne handicappede og socialt udsatte. Her er målet at systemet skal gøre sagsbehandlingen mere effektiv. 

Det er derfor i EG Team Onlines interesse, at få foretaget en forundersøgelse af det eksisterende marked, for at afgøre, om det er et marked det ville give mening for EG at gå ind på. Yderligere skal det undersøges hvorvidt det er muligt at implementere et system, der kan effektivisere sagsbehandling af voksne udsatte, i henhold til voksenudredningsmetoden. Der skal derfor analyseres, designes og implementeres en central del af systemet, der kan fungere som grundlag for en eventuel konstruktion af det fuldt funktionelle system.

Dette projekt omhandler udviklingen af Sensum Udred, med fokus på login og sagsåbning. I projektet vil der blive udviklet en prototype af et system, som skal kunne håndtere sagsåbning af en sag vedrørende voksenudredning, ud fra guidelines beskrevet i voksenudredningsmetoden. Til udviklingen af Sensum Udred vil der blive brugt en kombination af Unified Process og Scrum, med "Scrum-buts". Der vil altså blive brugt metoder og planlægningværktøjer fra begge projektstyrings-processer. Til udvikling af analyse og design af systemet vil Unified Process blive brugt, og modelleret og dokumenteret ved brug af UML. Systemet skal ikke kunne arbejde med andre systemer, da det var et krav, at det skulle være et enestående system. Det skal stadig kunne udføre den ønskede funktionalitet. Rapporten viderebygger på projektetableringsfasen og inceptionsfasen, som gruppen har været igennem. Denne rapport skrives på baggrund af elaborationsfasen, og de foregående faser, for at danne en endelig rapport over hele projektet.

\subsection{Problemstilling}

Firmaet EG Team Online (EG) overvejer I forbindelse med monopolbruddet indenfor IT-systemer til det offentlige, udviklingen af et system til voksenudredning, kaldet Sensum Udred. 
Inden udviklingen af dette system, skal der dog foretages nogle forundersøgelser, for at sikre at Sensum Udred er værd at udvikle. 

Det skal fra EG's side overvejes hvorvidt systemet giver mening at udvikle i forhold til dets anvendelse. Kan det implementeres således, at voksenudredningsmetoden ikke bare overholdes, men også effektiviseres nok til at udviklingen giver mening? Dette vil blive undersøgt ved at udvikle en prototype på et afgrænset område af systemet. Denne prototype skulle gerne vise et eksempel på en mere effektiv udførsel af det afgrænsede område, end det har været muligt ved tidligere løsninger, og samtidig overholder de foreskrevne standarder i voksenudredningsmetoden. Her er det blevet valgt at fokusere på processen omkring login og sagsåbning, altså begyndelsen af en udredning. 

Disse punkter formodes at kunne afklare hvorvidt udviklingen af Sensum Udred vil give mening, og om det vil være i stand til, på en holdbar måde, at effektivisere arbejdet for kommunerne. 


\paragraph{Problemformulering}\mbox{} \\
\textsl{Kan der udvikles et Sensum Udred med Unified Process, der effektiviserer sagsbehandlinger i henhold til voksenudredningsmetoden, som overholder persondatabeskyttelsesforordningen?}


\subsection{Formål og mål}
Formålet med projektet er, at skabe en forståelse for den proces der ligger bag at udvikle et større softwaresystem, helt fra at undersøge og forstå markedet, gjort i inceptions-fasen, til den egentlige udvikling og implementering af systemet.
Yderligere undersøges også krav fra andre instanser end kunden, i form af persondatabeskyttelsesforordningen. Dette vil bidrage til oplevelsen af, at arbejde under de forhold en organisation som EG Team Online arbejder under på større projekter.

Hele processen, komplet med kravspecifikation, analyse og design, implementering, og dokumentation gennemføres ved brug af Unified Process, med projektstyrings-elementer fra Scrum. Dette gøres for at give en dybere forståelse for det at arbejde efter en process-model, og derudover det at sortere i hvilke elementer man ønsker at benytte fra Scrum. 

\paragraph{De opsummerede mål:} \mbox{} \\
\begin{itemize}
\item Benytte Unified Process i kombination med Scrum til udarbejdelse af projektet og projektstyring.
\item Arbejde med, og forstå artefakter fra Unified Process, og benytte UML til at modellere disse.
\item Designe og implementere en prototype af gruppens udvalgte systemafgrænsning, dette iterativt og objektorienteret.
\item Designe og implementere en relationel database.
\end{itemize}

Målene er i overenstemmelse med de mål der er opsat, både for faget SI2-ORG og for 2. semester som helhed, beskrevet i Studiordningen kapitel 9.

\subsection{Motivation}

Gruppen vil gerne opnå et fagligt godt resultat. Dette forventes at kunne opnås ved at arbejde struktureret, med en arbejdskultur hvor alle medlemmer forventes at tage ansvar, og uddelegerede opgaver forventes at blive udarbejdet til de specificerede deadlines, medmindre andet er aftalt. 
En motivation for gruppen er at skabe et godt miljø til vidensdeling, hvor alle medlemmer har en god forståelse for alle aspekter i projektet. Gruppen vil opnå dette miljø i form af reviews, samt ved at præsentere uddelegeret arbejde for resten af projektgruppen, før dette arbejde indarbejdes endeligt i projektet. Her er målet og motivationen at gruppen i fællesskab vil gøre hinanden bedre.
Derudover er en generel interesse i de faglige elementer i projektet også en motivationsfaktor. 

\subsection{Målgruppe}
Målgruppen for rapporten er alle med en interesse i organisationsorienteret softwareudvikling, og som ønsker et indblik i de processer og teknikker der ligger bag udviklingen af en større softwareløsning.
Målgruppen indbefatter også alle der ønsker indblik i tankerne bag implementeringen af gruppens afgrænsning af Sensum Udred.
Yderligere er rapporten udarbejdet til alle med en akademisk interesse i projektet, hvad end det er i de benyttede metoder, eller blot til evaluering.
\newpage
\end{document}
