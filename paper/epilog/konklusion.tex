\documentclass[../main.tex]{subfiles}

\begin{document}
\section{Konklusion}

Projektet Sensum Udred har været med til at skabe en forståelse for den store process i at udvikle et kommunalt system, som i vores tilfælde var afgrænset en del af Sensum Udred. Forståelsen for dette er blevet skabt igennem Inceptionsfasen, hvor projektets afgrænsing blev fastlagt, og derefter forståelsen for hvordan implementeringen for Sensum Udred sættes igang ved hjælp af Elaborationsfasen fra Unified Process. Ved hjælp af Unified Process er Sensum Udred blevet udviklet iterativt, og har været med til at skabe et konkret Sensum Udred, der opfylder funktionaliteten fra projektets formål. Med hjælp fra Scrum, som projektstyringsværktøj er det blevet mere forståeligt at bruge en process-model til udviklingen af software. Scrum har været med til at skabe præcise resultater, og sørget for at fokus på projektets brugsmønstre er blevet holdt. Ved at bruge Scrum i Elaborationsfasen, har det været muligt at overholde en konkret tidsplan for projektet. Udover dette har det styrket samarbejdet som team, da det pointerer positive elementer og negative ting igennem sprint retrospektiv. Dette har optimeret projektet og projektstyringen. 
Brugen af Unified Process har gjort det muligt, at modellere diagrammer i UML med UP's artefakter. Ved at bruge artefaktet brugsmønsterrealisering, er det blev muligt at oprette domænemodeller eller analysediagrammer, som senere kan bruges til designmodellen. På den måde har artefakterne været med til at styrke grundlaget for at designe et stærkt Sensum Udred. Ud fra målet om at designe og implementere en prototype ud fra systemafgrænsingen, er dette blevet gjort, og skabte en grundskitse for Sensum Udred, som der var god at tage udgangspunkt i. På den måde sikredes det, at størstedelen af klasser var med fra start, og den lagdelte struktur blev holdt for øje. Sensum Udred skulle kunn elagre data sikkert. Et af målene for projektet var at designe en relationel database og derefter implementere den i kodebasen. Ved hjælp fra designklassediagrammet, som havde beskrivelser for associationer med mellem klasserne, og ansvarfordeling, var det nemt at oprette en relationel database. Dataen der skulle gemmes i den relationelle database var allerede modelleret. Derved blev implementeringsarbejdet også gjort nemmere. Alt i alt har det været et lærerigt projekt, der har været med til at give indblik i de processer der ligger bag at udvikle software systemer som en organisation, og den mængde arbejde og overvejelser der hører til, samt givet redskaber med til at kunne udvikle komplekse systemer, som Sensum Udred.

\end{document}
