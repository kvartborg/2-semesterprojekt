\documentclass[../../main.tex]{subfiles}

\begin{document}
\section{Projektforslag}
\subsection{Introduktion}
EG team online ønsker at bryde monopolet, som KMD tidligere har haft.
Dette gøres ved at udvikle et nyt Sensum Udred, der skal benyttes til voksenudredning for kommuner.
Der skal i dette forløb redegøres for hvorvidt EG Team Online’s løsning er en god ide.

\subsection{Problemstilling}
EG Team Online overvejer at starte udviklingen af et digitalt sagsbehandlingssystem
til voksenudredning ved navn Sensum udred, i forbindelse med det såkaldte
monopolbrud i den offentlige sektor. Inden udviklingen startes skal det undersøges
det er et godt marked at gå ind i. Dette gøres ved at lave en forundersøgelse, som
blandt andet indeholder en business case. Samtidigt med at systemet skal give
forretningsmæssigt mening skal systemet også forbedre sagsbehandlingsforløbet.
Gruppen vil derfor undersøge om Sensum Udreg kan forbedre og standardisere
sagsbehandlingsforløbet.
Kommunen underlagt høje krav i forhold til håndteringen personfølsomme
data (EU-Persondatabeskyttelsesforordning), og det er derfor vigtigt når man
udvikler et offentligt system. Gruppen vil derfor lægge fokus på at sikre
personfølsomme data i projektet. Kan implementeringen forbedre og standardisere
sagsbehandlingsforløbet? Kan Sensum Udred sikrer personfølsomme data?

\subsection{Formål og mål}
Projektets formål er, at benytte Unified Process til at analysere EG Team Onlines
involvering i udviklingen af Sensum Udred, hvordan implementeringen af
Sensum vil forbedre og standardisere arbejdsgangen på tværs af kommunerne, og
hvordan persondatabeskyttelsesforordningen skal implementeres, således at systemet
lever op til den nuværende lovgivning.

\paragraph{Mål}
\begin{itemize}
  \item Bruge Unified Process til udarbejdelse af projektet.
  \item Designe software systemet iterativt og objektorienteret.
  \item Designe og implementere en relationel database.
\end{itemize}

\subsection{Motivation}
Gruppens motivation er at opnå en god forståelse hvordan Unified Process kan
benyttes til projektstyringen af Sensum Udred, samt at udrede en god analyse
i inception, elaborerings fasen og implementering af den valgte problemstilling.
Gruppen vil gerne opnå et godt resultat.
Måden hvorpå gruppen forventer at opnå dette mål er ved at arbejde struktureret,
alle medlemmer forventes at tage ansvar, og uddelegerede opgaver forventes at
blive udarbejdet til de specificerede deadlines, medmindre andet er aftalt.
Et fokusområde for gruppen er at skabe et godt miljø til vidensdeling, så alle
medlemmer har en god forståelse for alle aspekter i projektet.
Gruppen vil opnå dette miljø i form af at reviews, samt ved at præsentere
uddelegeret arbejde for resten af projektgruppen, før dette arbejde indarbejdes
endeligt i projektet.

\subsection{Forventede resultater}
Vi forventer at aflevere en rapport som svarer på vores problemstilling, samt en
prototype implementering, der følger Projekthåndbogens fremgangsmåde.

\subsection{Foreløbig tidsplan}
\begin{center}
  \begin{tabular}{ c | c c c }
    Fase & Uge & Faseindhold og Milepæle & Tidspunkt/frist \\
    \hline
    Fase & Uge & Faseindhold og Milepæle & Tidspunkt/frist \\
  \end{tabular}
\end{center}

\subsection{Samarbejdsaftale}
stk. 1 - Alle meninger, holdninger, ytringer, idéer mm. som fremlægges i
gruppen, skal vægtes lige.\\

stk. 2 - Alle medlemmer af gruppen får tildelt løbende ansvarsområder, som
forventes udført til respektive deadlines. Hvis ikke disse deadlines overholdes,
er de resterende medlemmer af gruppen obligerede til at henstille vedkommendes
opmærksomhed på hans eller hendes forsømmelser og bede vedkommende
overholde fremtidige deadlines.\\

stk. 3 - Bliver projekt-relateret arbejde ikke færdiggjort i indlagt skoletid,
skal personen med denne opgave inden for sit ansvarsområde færdiggøre
dette i sin fritid. Dette skal ske uden benægtelse.
Det er ok at spørge andre i gruppen om hjælp.\\

stk. 4 - Har personen med ufærdiggjort projekt-relateret arbejde, som
beskrevet i, stk 3, med god grund ikke mulighed for at færdiggøre dette
fyldestgørende, har denne person et ansvar for at informere resten af gruppen
om dette, tidsnok til at andre gruppemedlemmer kan færdiggøre denne opgave.
Overholdes denne regel ikke, kan konsekvenser indebære en samtale med vejleder.\\

stk. 5 - Opstår en konflikt i gruppen, skal denne løses diplomatisk med samtale
som medium. Det er derfor et behov at gruppemedlemmer, som føler sig uretfærdigt
behandlet el.lign. straks siger dette til resten af gruppen.\\

stk. 6 - Er et gruppemedlem nødsaget til at sygemelde sig, eller på anden vis
er udeblivende, skal en rimelig grund oplyses til mindst ét gruppemedlem, som
herefter har til ansvar at videreformidle dette til resten af gruppen.\\

stk. 7 - Alle gruppemedlemmer har ansvar for at møde til tiden, være
veludhvilede, friske, motiverede og klar til at arbejde på semesterprojektet.
Overskrides dette, henvises til stk. 5.\\

stk. 8 - Ambitionsniveauet for gruppen er højt, og der forventes et produkt af
høj kvalitet.\\

stk. 9 - I tilfælde af en konflikt der ikke kan løses og gruppen splittes vil
al arbejde / viden op til opdelingstidspunktet blive delt.
Efter bruddet vil der ikke være fortsat samarbejde  mellem grupperne.\\

stk. 10 - Ved start af gruppens møder gennemgås alle ændringer i projektet siden
sidste møde, til en grad så hele gruppen har forståelse for projektets tilstand.\\

stk. 11 - Husk at have det sjovt og fejre at delmål opnås.\\

stk. 12 - Forventede arbejdstimer ugentligt er 9 fælles, 4 timer forberedelse.
Gruppen regner med at mødes hver tirsdag og onsdag, vi vil benytte fredag
som en buffer.\\

stk. 13 - I tilfælde af uventet overarbejde vil alle gruppemedlemmer bidrage til
brandslukning. Dette gælder alle dage også i weekenden.\\


\subsection{Vejlederkontrakt}

\subsubsection{Vejledning mellem møder}
Ved kommunikation over Mail svares der hurtigst muligt, både fra vejleder og gruppes side.
I akutte tilfælde, kan email benyttes, ellers ventes der til et vejledermøde.

\subsubsection{Vejledermøde}
Som udgangspunkt er der ét vejledermøde om ugen.

Gruppen afleverer/sender en agenda til vejleder senest 24 timer før et møde.
Hvis der ikke sendes en agenda, har vejleder ikke pligt til at møde op. Gruppen
skal sende en aflysnings email, hvis gruppen ikke ønsker / ikke har mulighed
for et møde.
Svarer gruppen ikke, eller har der ikke været meningsfuld kontakt inden
for 2 uger, vil vejleder tage affære, som vejleder finder det passende.

Gruppen skal have en ansvarlig for kommunikationen mellem vejleder og gruppen.
Dette er for at sikre at alle i gruppen ikke regner med at en anden gør dette.

Gruppen vælger til hvert møde en referent og en ordfører.
Ordføreren sørger for at gruppen få svar på alle deres spørgsmål på den
givne tid, og referenten sørger for at skrive noter fra mødet.
Efter hvert møde sendes referatet til alle i gruppen +  vejleder.
Vejleder godkender at det skrevne er korrekt. Referatet sendes helst
rundt samme dag, men senest 24 timer efter mødet. Husk at ordføreren ikke
taler for gruppen, men kun sørger for at alle emner når at blive
omtalt i den angivne mødetid.

\end{document}
